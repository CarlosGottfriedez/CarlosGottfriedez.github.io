\documentclass[]{article}
\usepackage[T1]{fontenc}
\usepackage{lmodern}
\usepackage{amssymb,amsmath}
\usepackage{ifxetex,ifluatex}
\usepackage{fixltx2e} % provides \textsubscript
% use upquote if available, for straight quotes in verbatim environments
\IfFileExists{upquote.sty}{\usepackage{upquote}}{}
\ifnum 0\ifxetex 1\fi\ifluatex 1\fi=0 % if pdftex
  \usepackage[utf8]{inputenc}
\else % if luatex or xelatex
  \ifxetex
    \usepackage{mathspec}
    \usepackage{xltxtra,xunicode}
  \else
    \usepackage{fontspec}
  \fi
  \defaultfontfeatures{Mapping=tex-text,Scale=MatchLowercase}
  \newcommand{\euro}{€}
\fi
% use microtype if available
\IfFileExists{microtype.sty}{\usepackage{microtype}}{}
\usepackage[margin=1in]{geometry}
\ifxetex
  \usepackage[setpagesize=false, % page size defined by xetex
              unicode=false, % unicode breaks when used with xetex
              xetex]{hyperref}
\else
  \usepackage[unicode=true]{hyperref}
\fi
\hypersetup{breaklinks=true,
            bookmarks=true,
            pdfauthor={},
            pdftitle={DatenBasis},
            colorlinks=true,
            citecolor=blue,
            urlcolor=blue,
            linkcolor=magenta,
            pdfborder={0 0 0}}
\urlstyle{same}  % don't use monospace font for urls
\setlength{\parindent}{0pt}
\setlength{\parskip}{6pt plus 2pt minus 1pt}
\setlength{\emergencystretch}{3em}  % prevent overfull lines
\setcounter{secnumdepth}{5}

%%% Change title format to be more compact
\usepackage{titling}
\setlength{\droptitle}{-2em}
  \title{DatenBasis}
  \pretitle{\vspace{\droptitle}\centering\huge}
  \posttitle{\par}
  \author{}
  \preauthor{}\postauthor{}
  \date{}
  \predate{}\postdate{}




\begin{document}

\maketitle


{
\hypersetup{linkcolor=black}
\setcounter{tocdepth}{2}
\tableofcontents
}
\section{Daten als Grundlage nicht
Texte}\label{daten-als-grundlage-nicht-texte}

Jeder Arzt produziert Daten in dem er Diagnosen stellt oder allgemeine
Einschätzungen trifft beispielsweise wenn es um den weiteren
Behandlungsplan geht. Diese Einschätzungen werden mündlich oder
schrifftlich in Form z.B. eines Befundes festgehalten. Diese Befunde
werden von anderen Kollegen, vom Patienten selbst oder von Dritten
gelesen und beurteilt. Im Verlauf einer Behandlung werden immer mehr
Daten in Form von Texten produziert, die schlussendlich wieder von
Menschen interpretiert und verstanden werden müssen. Wäre es nicht
einfacher, wenn der Arzt als eine Instanz angesehen werden kann die bei
entsprechender Eignung von einer endlichen Menge an Untersuchungen
spezielle Daten erheben kann. So kann man sich, wie ich denke, auf ein
Datenschemata speziell für jede Untersuchungen die es in der Medizin
gibt und es jemals geben wird einigen und dies im Laufe der Zeit auch
anpassen. Beispielsweise kann man für eine Röntgenuntersuchung eine
bestimmte endliche Menge an Variabeln (Attributen) definieren die bei
der entsprechenden Untersuchung verifiziert werden ``können''. Auf
dieser Datengrundlage können dann Befunde, Analysen, Dokumentationen,
Berichte erstellt werden und eventuell automatisiert werden. So ist
nicht mehr der von Mensch geschriebene Text die Grundlage sondern die
Daten die vom Menschen oder wem auch immer erzeugt wurden

\end{document}
